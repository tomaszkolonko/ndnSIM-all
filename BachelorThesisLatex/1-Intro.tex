\chapter{Introduction}

In the past years content production and dissemination have both drastically increased. That has led among others to the wide use of Content Distribution Networks (CDN) that tackled the problem of distributing content more efficient and without slowing down the main servers. Not only the amount of data has changed but also the overall mobility has increased exponentially. A commuter expects to stream full HD movies on her cell phone while traveling in trains to work at high speeds. Mobile IP tried to solve this, but the main problem of an secure and always active host-to-host connection still remains.

\section{Motivation}

NDN is an implementation of ICN and tries to solve the above mentioned problems through a paradigm shift in networking. It is build upon the same innovative concepts as other ICN implementations. These are among others the named content and the routing by that name as in-network caching of data. The communication is also based on the Interest / Data model instead of maintaining an end-to-end connection at all time. This makes NDN with ndnSIM (it's simulator) very appealing for mobile ad-hoc environments. There are a few forwarding strategies already implemented within ndnSIM but as of ndnSIM version 2.0 they all work through faces and point to point wire connections. That makes it very difficult to use in a wireless fashion, since all net devices have the same face id's.

\section{Study Subject}

The goal of this theses is to implement a basic forwarding strategy that can be used in a mobile ad-hoc environment which then can be used as a baseline. The basic implementation of the strategy should then be reiterated into a improved strategy with better bandwidth efficiency. This goal can be divided into four subtasks. First a new forwarding mechanism should be implemented by not only using faces but also a further unique identifier, like a mac address. FIB and PIT entries need therefore to be extended in order to hold additional information about previous and next nodes of both interests and data. The second task of the thesis is around implementing a forwarding strategy that decides how to route the interests, once the flooding is done and the FIB and PIT entries have been populated with the correct mac addresses. The third part is applying both mentioned mechanisms into a mobile ad-hoc scenario where nodes move around leave and enter the observed premises. The fourth part is to achieve a multipath transmission where specific nodes simultaneously send out interests and receive data.

\section{Outline}

The remainder of this thesis is organized as follows. Chapter 2 explains shortly why a paradigm shift takes place from host-based networking to content-centric networking, explains what CCN is and why it could outperform the TCP/IP model. Chapter 3 explains ndnSIM and it's current implementation of the forwarding mechanism and the strategy as of version 2.0. Chapter 4 shows the steps taken to implement a new forwarding based on not only faces but also on macs and discusses the multipath approach taken. In Chapter 5 the results are evaluated and discussed. Finally a conclusion of the thesis is offered in Chapter 6. (TODO: take out the appendix or write here something).
