\chapter{Introduction}

In the past years content production and dissemination have both drastically increased. That has led among others to the wide use of Content Distribution Networks (CDN) that tackled the problem of distributing content more efficient and without slowing down main servers. Not only the amount of data has changed, but also the overall mobility has increased exponentially. A commuter on his way to work expects to stream full HD movies on his cell phone or at least listen to Spotify while traveling by train at high speeds. Mobile IP was attempt to solve some of the issues with high mobility, but the main problem of secure and always active host-to-host connections still remains, and is one of the reasons why the need for a new network architecture becomes more and more pressing.

\section{Motivation}

Vehicular Ad-Hoc Networks (VANETs) are mobile entities that create spontaneously wireless networks for the exchange of data. This exchange can be for a very short or a prolonged time. It depends on a broad variety of applications from very basic information about a moving car in front or rather weather information from a central server. Information dissemination of messages in certain areas play an important role to VANETs and question arise how to best solve the issues that arise with no static topology, fast moving entities like cars, drones, bikes or even trains that can be used for propagation of data over long distances to place unreachable by other means. New forwarding strategies need to be implemented in order to get information about possible content sources and how data can reach fast moving entities. Problems like intermittent links and no static host to host connections play an important role among other problems like security and privacy, content caching and efficiency.

Named Data Networking (NDN) is an implementation of Information-Centric Networking (ICN) and tries to solve the above mentioned problems through a paradigm. It is build upon the same innovative concepts as other ICN implementations. These are among others the named content and the routing by that name as in-network caching of data. This is in contrast to the current Internet architecture that is based on host to host communication. 

This makes NDN with ndnSIM (it's simulator) very appealing for mobile ad-hoc environments with no static topology whatsoever. New forwarding strategies need to be implemented in order to get information about possible content sources and how data can reach fast moving entities. Problems like intermittent links and no static host to host connections play an important role among other problems like 

\section{Study Subject}

The goal of this thesis is to implement a forwarding strategy that can be used in a mobile ad-hoc environment and tested against different scenarios for performance metrics. The implementation will be done in NS-3 and ndnSIM which is a simulator based in C++. First some research is needed to find out what has been implemented already in ndnSIM. If there are some strategies already, the main goal will be to implement a forwarding strategy that will outperform the default strategy. The scenario will be based on VANETs. Nodes will be used as mobile entities that move and interact witch each other. The scenario will start with no prior knowledge of the topology and as data passes through the network routes and nodes will be discovered. This information will be used to further improve the dissemination of information.

\section{Outline}

The remainder of this thesis is organized as follows. Chapter 2 explains shortly why a paradigm shift takes place from host-based networking to content-centric networking, explains what Content-Centric Networking (CCN) is and why it could outperform the TCP/IP model. It will contain and go into related work already done on this subject. Chapter 3 explains ndnSIM and it's current implementation of the forwarding mechanism and the strategy as of version 2.0. In chapter 4 the design and the implementation of the forwarding strategy is described. In addition, the modifications that have been performed on ndnSIM together with the algorithm for the forwarding strategy are presented. In Chapter 5 the results are evaluated and discussed. Finally a conclusion of the thesis is offered in Chapter 6.
