\chapter{Conclusion}


\section{Summary and Conclusion}

The goal of this thesis was to develop a new forwarding strategies for VANETs in ndnSIM version 2.0. It has been shown that routing can be done through MAC addresses over WiFi 802.11a without utilizing the faces originally used by ndnSIM. A new strategy was implemented and tested against the existing multicast strategy. On all the measured metrics the new implementation outperformed the current base implementation in ndnSIM.
Comparing the percentage of satisfied interests obtained in the static scenario by both implementations, a significant improvement could be observed. Taking also the number of interests and data packages into account, it becomes clear that the performance has been increased manyfold with reduced strain on the bandwidth of the network. The constant latency in the new implementation shows that over time there is no significant accumulation of unsatisfied interests floating inside the network. The amount of satisfied interests have been more than tripled in the static scenario and more than doubled in the dynamic one.

\vspace{5mm} %5mm vertical space

The results on the dynamic scenario are also quite good and promising compared to the default implementation in ndnSIM. The percentage of satisfied interests in the new implementation could be drastically increased over all simulation times. It starts lower than on the static scenario but that was expected since the topology is more complex and moving. As the satisfied interests accumulate the percentage goes steadily up. The amount of interests send out to the network and being satisfied has been also improved. The number of interests and data processed by the network are less in the dynamic scenario compared to the static one. That also was expected, because the size of the network is not correlated with the amount of data transferred but the consumer's sending frequency and successful forwarding of the network. The reason for the lower interest and data rates on the default implementation is congestion that increases exponentially with more nodes and more net devices, whereas on the improved implementation the main problem are the preferred static routes (FIB next hops). The latency is very similar to the static scenario and has been improved on the new implementation significantly (low and constant).

\section{Future Work}

CCN and it's implementation in ndnSIM are still in the early beginnings although already several years into research. As devices get more in number and become highly mobile there are many scenarios that need further investigation.

\subsection{Static 8 nodes scenario}

For the static 8 nodes scenario interesting future work could be to improve on dynamic route selection. Therefore not arbitrary increment or decrementing the cost of the FIB next hops but relying on some metric and feedback from the network. Lower cost of the next hops is not merely to select the cheapest route relative to the other next hops. It has a direct impact how reliable and fast the interest can be forwarded upstream. Therefore keeping the cost as low as possible is important for a performant network. NACKs (negative acknowledgments) could be implemented reducing the time between retransmissions, as they have in future versions of ndnSIM. Adding 100 or 200 nodes to the current scenario would give important insight into scalability of the used implementation, especially since scalability is an important feature and advantage of the CCN architecture. The achieved improvements could be potentially be multiplied by introducing a CS to the scenario. Since the main focus of this thesis was the forwarding algorithm, the CS implementation was not included. Although it was not needed since only one consumer was used requesting different packages every time. CS and scalability become more interesting when several consumers and producers are added to the simulation, and the consumer request the same data.

\subsection{Dynamic 8 nodes scenario}

For the dynamic 16 node scenario introducing a timer for the FIB entries should result in more sophisticated routing. If a FIB entry has not been used for a certain amount of time it should be deleted forcing flooding and therefore get current information about the next neighbour (and consequently about the topology). That problem was described in figure \ref{ig:scenario2extended}. That would lead to dynamically find new routes when a time out occurs. It is similarly implemented in the PIT entries but not yet in the FIB entries. Negative acknowledgements could also be used to signal that an interest could not be forwarded to the node present in the FIB's next hop. Deleting in that case the FIB would also lead to refreshing the information about the topology. In such a case latency should be reduced significantly since the routes would be constantly updated. Ratio and throughput would be also increased.






