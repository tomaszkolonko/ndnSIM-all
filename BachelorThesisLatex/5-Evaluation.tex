\chapter{Evaluation}

The evaluation is conducted on several independent scenarios. As mentioned in chapter 1 the thesis relies on four subtasks that need to be solved. First the forwarding should rely on MAC addresses of the WiFi network interface (also called net devices) and not the different faces. The second subtask is to implement a forwarding strategy that can be used with the MAC addresses. The third part is to add mobility to the intermediate nodes while the fourth part is to improve on the forwarding strategy. Every scenario is evaluated and compared to the ndnSIM's multicast strategy \texttt{nfd::fw::MulticastStrategy}, which has been implemented in ndnSIM version 2.0 as basic multicast strategy. The thesis will refer to this multicast strategy as the base strategy or only base. It basically receives an interest and forwards it to all available faces of the node except the receiving one. The newly implemented forwarding strategy will often be referred to as the new strategy or the improved strategy.

To achieve comparability the scenario and trace-files are copied into a clean base installation of ndnSIM version 2.0. All parameters are kept equal. Then it is run once on the clean installation and once on the improved one. In the first scenario the implementation of the mac addresses and the forwarding mechanism is tested and evaluated. In the second scenario mobility is introduced with changes to the strategy. The different scenarios are tested for interest/data ratio, latency, congestion and number of distinct routes, although distinct routes do not exist in the already existent multicast strategy. It is introduced as part of the thesis for the new implementation.

\section{Evaluation environment}

All tests are run on the ns-3 network simulator with ndnSIM version 2.0. Different trace files are used for the two distinct scenarios. NS2 Mobility Helper uses these trace files and adds movement to the nodes. As Wifi standard 802.11a was used, which means each node has a transmission range of approximately 120 meters in open space \cite{wifi80211a} . For the propagation delay \texttt{ns3::ConstantSpeedPropagationDelayModel} was used and for the propagation loss \texttt{ns3::ThreeLogDistancePropagationLossModel} and \texttt{ns3::NakamiPropagationLossModel} were used. Interest lifetime is set to 4 seconds and the retransmission timeout to 500ms. Data packet size is 1200kb.

\section{Results for a static 8 nodes scenario}

The first scenario is a static 8 node scenario with one consumer on the left and one producer on the right. 6 intermediate nodes are placed in between as seen in figure \ref{fig:scenario1}. Every node has 3 net devices with distinct MAC addresses. They allow on one hand to send and receive simultaneously, while extending on the multi-path idea. The paths are determined by the FIB entries that have been configured with MAC addresses from data being forwarded downstream. Therefore a first segment of an interest can be requested through the first net device. The second segment through the second net device and the third segment through the third net device. The fourth segement would be sent again through the first net device. That is achieved by a static counter and a modulo operation leading to different routes.

\begin{figure}[H]
  \centering
  \includegraphics[scale=0.5]{chapter-5/scenario1}
  \caption{Basic static scenario with 8 nodes, 1 consumer and 1 producer}
  \label{fig:scenario1}
\end{figure}

The runs have been conducted for 100, 200, 300 and 600 seconds each. The consumer was configured to sent out interests at a frequency of 3 interests per second. For the ratio, received data over send interests only distinct packages were counted. Retransmissions were ignored. The average latency was measured in nanoseconds, as the time difference of the interest leaving the consumer and the corresponding data package being received at the consumer, added together and divided by all successful interest requests.

\begin{figure}[H]
  \centering
  \includegraphics[scale=0.6]{chapter-5/staticS1ratio}
  \caption{Satisfaction ratio at different simulation times for both implementations}
  \label{fig:staticS1ratio}
\end{figure}

Figure \ref{fig:staticS1ratio} compares the ratio of satisfied interests (interests send / data received back) over simulation time. A better ratio has been observed for the new implementation at all simulation times. The difference is biggest at a run time of 100 seconds. At a run time of 600 seconds the difference has been reduced to 0.171. That is due to the amount of interests and data being in transition. As the number of satisfied interest constantly rises over time, the unsatisfied interest will not keep getting more since retransmissions will be made after new interests are introduced to the network.

Figure \ref{fig:staticS1iod} shows that the ratio alone (as shown in figure \ref{fig:staticS1ratio}) does not make any statement about the amount of the interests sent towards a potential content producer and the amount of received data. With the base implementation there is not much gain with increased simulation time whereas with the new implementation the gain nearly doubles. The reason for that observation is explained with congestion. Broadcasting the interest at each node leads to an exponential increase in the packages being retransmitted within the network leading to congestion, loss of packets and retransmissions. The new implementation floods the network only with the first interest and uses configured routes for all further interests leading to much less traffic. Retransmissions have been observed at the consumer and although the new implementation has fewer retransmissions (around 15 percent) it is not very significant and they happen on a much higher transmission rate. It can be expected that the retransmission rate is strongly correlated with the retransmission timer in the consumer and the lifetime of the interest itself.

\begin{figure}[H]
  \centering
  \includegraphics[scale=0.6]{chapter-5/staticS1iod}
  \caption{Number of sent and received packages over simulation time}
  \label{fig:staticS1iod}
\end{figure}

Figure \ref{fig:staticS1latency} shows how the latency changes with the simulation time and chosen implementation. As expected from \ref{fig:staticS1iod} the latency increases significantly with congestion and retransmissions in the base implementation of the multicast strategy. In the first 100 seconds it already reaches 40 seconds which multiplies by 3 at 600 seconds. The new implementation has a much lower latency, starting with around 11 seconds for the first 100 seconds of the simulation. Then increases slightly till around 15 seconds for 600 seconds of simulation. That also shows that congestion and retransmission stay constant for the new implementation.

\begin{figure}[H]
  \centering
  \includegraphics[scale=0.6]{chapter-5/staticS1latency}
  \caption{Latency of both implementations over simulation time}
  \label{fig:staticS1latency}
\end{figure}

The implementation has been tested with overhearing of data coming back and adding the route to the FIB entries but with a higher cost. If the data was intended for the receiving node, the node added or updated the FIB with the new information and attached cost 111 to this specific next hop. The number 111 was randomly chosen and represents a relative low cost. If the data was only overheard (unsolicited) a FIB entry was updated or created with cost 222 instead of ignored and dropped immediately to save resources. The results were slightly worse than the above proposed implementation, but still much better than the base one.

\vspace{5mm} %5mm vertical space

Since one to three new FIB entries were added to every node, an additional approach was to test, if instead of broadcasting or forwarding the interest to only one next hop, better results could be achieved by forwarding the interest to two or three distinct next hops. This yielded again worse results and lead to the above proposed implementation.

\vspace{5mm} %5mm vertical space

One problem with more than one valid FIB entry of next hops was, that the main routes remained the same over the simulation. Instead of alternating between possible next hops only the first one was chosen and the interest forwarded to. Incrementing each cost by one and sorting the FIB's next hops did increase the number of different routes taken. The next hops within the FIB entry were selected in alternating fashion. Unfortunately that did not improve the overall performance. It decreased it despite having more routes to chose from. The reason for this behaviour is how ndnSIM translates FIB next hops cost into transmission time. The lower the cost the more reliable and faster the interest is forwarded. For example setting the cost to 50 after reaching 150 while incrementing all forwarded FIB next hops by one led to significantly better results then looping the cost from 250 to 150. Since that is true for both scenarios equally, it has not been further investigated.

\section{Results for dynamic 16 nodes scenario}

For the dynamic scenario 8 further nodes were introduced. All 14 nodes have been randomly scattered between the consumer and the producer. Random movement was added to the trace file by changing direction and speed of every intermediate node at 5 second intervals. The net devices remain to be three. Retransmission time is set to 500 milliseconds, while the interest lifetime also remains at 4 seconds. 

Figure \ref{fig:scenario2} shows the topology at the beginning of the simulation. The consumer and producer are marked in the figure, and remain static. Only intermediate nodes move in a random manner at different speeds. The distance between consumer and producer as been increased to about 450 meters.

\begin{figure}[H]
  \centering
  \includegraphics[scale=0.4]{chapter-5/scenario2}
  \caption{Ad-hoc scenario with 16 nodes and movement, 1 consumer and 1 producer}
  \label{fig:scenario2}
\end{figure}

For the static scenario it was sufficient to keep the configured routes constant and not alternate through the intermediate nodes quite that often. On this scenario though, by moving intermediate nodes, the strategy may possibly lead to losing the connection altogether (if the nodes move too far away). The base implementation of ndnSIM does not have this problem since it has no specific routes to follow and basically broadcasts the interest all over again using the next best node available. The trace file was randomly generated in order to give a realistic scenario and the above problem can be seen around simulation time: 200 seconds in figure \ref{ig:scenario2extended}. This problem is also noticed in the numeric data.

\begin{figure}[H]
  \centering
  \includegraphics[scale=0.4]{chapter-5/scenario2extended}
  \caption{Node 10 is one of the producer preferred nodes while node 14 is one of the consumer preferred nodes.}
  \label{fig:scenario2extended}
\end{figure}

To counter the aforementioned problem, the interest is forwarded to three FIB next hops (instead of only one) and making the strategy multicast to three nodes upstream the default. Sending the interest to three MAC addresses increases robustness at the cost of generating more traffic, but overall leading to much better results without congesting the network.

Figure \ref{fig:scenario2ratio} shows the ratio of send interest packages over satisfied ones. The new implementation starts from 71.8\% satisfaction rate for the first 100 seconds of simulation and goes up to almost 90\% for the 600 seconds simulation. The ndnSIM base implementation starts lower, but as time passes by, the interests get satisfied through broadcasting and the satisfaction rate rises up to 74.3\%.

\begin{figure}[H]
  \centering
  \includegraphics[scale=0.6]{chapter-5/scenario2ratio}
  \caption{Satisfaction ratio at different simulation times for both implementations with mobility}
  \label{fig:scenario2ratio}
\end{figure}

Like in the static scenario it is important to look at the overall throughput and set the satisfaction ratio into context of send and satisfied interests. Figure \ref{fig:scenario2iod} shows the base implementation in brown colour. The dark bar shows the interests send upstream towards a potential content producer, while the lighter bar shows the satisfied interests. The ratio obtained in \ref{fig:scenario2ratio} is seen once again by comparing the bars of one colour per simulation time to each other. The new implementation has not only a better ratio but also is able to send and receive more data. The reason is that the nodes are targeted specifically by the mac address and don't congest the network even if send to three upstream addresses.
The new implementation sends 39 interests out and receives 28 data that satisfy the interests in the first 100 seconds of the simulation. The next 100 seconds roughly doubles the interests being sent and satisfied, whereas the next 100 seconds result in a decrease of newly satisfied interests. That problem was described above and can be seen figure \ref{ig:scenario2extended}. Different trace-file led to different results and whenever a clustering of the intermediate nodes was observed the successfully sent out and satisfied interests also slowed down. The clustering issue is also a problem for the ndnSIM base implementation strategy, but preferred nodes by FIB entries only pose a problem to the new implementation that needs further attention.

\begin{figure}[H]
  \centering
  \includegraphics[scale=0.6]{chapter-5/scenario2iod}
  \caption{Number of sent and received packages over simulation time}
  \label{fig:scenario2iod}
\end{figure}

Figure \ref{fig:scenario2latency} shows the observed latency for both implementations. As expected the latency is significantly better with the new implementation, although it is longer, due to increased number of nodes and more distance between the consumer and the producer. As the unsatisfied interests are retransmitted the latency increases over time.

\begin{figure}[H]
  \centering
  \includegraphics[scale=0.6]{chapter-5/scenario2latency}
  \caption{Latency of both implementations over simulation time}
  \label{fig:scenario2latency}
\end{figure}

In the static and dynamic scenarios the latency time is around 20 seconds, which is too high for real time applications. But can be used well in background processes, like downloading maps, weather forecast or entertainment media.

As an attempt to solve the problem mentioned in figure \ref{fig:scenario2extended} overhearing of unsolicited data was implemented. Different combinations have been tried with overhearing and dropping of unsolicited data. Overheard data could be used to add new FIB next hops of potential next hops otherwise not known and then be dropped or processed further. Overhearing data did yield many more next hops within the FIB entry, but didn't result in better performance. Incrementing the cost at every successful transmission led to overheard FIB entries having similar costs like the originally intended routes mixing them up. This also led to worse results and will need further investigation into this topic on how exactly increment and decrement the cost according to some defined metric and how to separate the overheard routes from the originally intended ones.


